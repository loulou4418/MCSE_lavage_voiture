\vspace*{4cm}
\section*{Résumé}
La Méthodologie de Conception de Systèmes Electroniques est une discipline essentielle lorsqu'il s'agit, dans le cadre d'une future carrière d'ingénieur, de réaliser un produit répondant aux besoins d'un client. 
Et ce domaine est d'autant plus indispensable dans un monde où les systèmes numériques sont de plus en plus complexes et leur conceptions demandent une bonne structuration et une réelle méthodologie. 
Ce rapport s'inscrit dans la présentation d'une conception d'un système de lavage de voiture par l'utilisation d'une méthode précise, applicable dans un ensemble de secteurs d'activités comme l'automobile, le médical, l'aéronautique, maritime, que ce soit dans le civil ou le militaire.
Il s'agit de concevoir un système simplifié de pilotage d'un portique de lavage de voiture. 
Ainsi, et afin de simplifier l'exercice, on ne considère que le rouleau horizontal du portique.

\newpage



\section{Introduction}

\subsection{Contextualisation}
La formation ETN (Électronique et technologies numériques) offerte par l'école polytechnique de l'Université de Nantes propose d'aborder diverses branches de l'électronique, du traitement du signal au systèmes à microprocesseur en passant par l'électronique analogique des hautes-fréquences. 
Cet ensemble de domaines techniques nécessite des compétences en matière de méthodologie de conception. Ce rapport s'inscrit dans la conception d'un appareil de marquage routier avec la méthode MCSE. 
La méthode MCSE (Méthode de conception des systèmes électroniques), née à Ireste par l'impulsion de Jean-Paul Calvez, cette méthode a été implantée au sein d'un outil nommée CoFluent rachetée par Intel\mbox{\textregistered } depuis 2011. Cette méthode fait désormais partie de la culture de la formation et constitue l'outil de conception premier de l'ingénieur ETN.\\
Ce rapport se décompose en diverses parties. 
Il s'agira dans un premier temps de rappeler le cahier des charges de la conception \hl{.......} %TODO
Pour l'ensemble de ce rapport, les diverses phases de spécifications et conceptions s'appuient sur les deux ouvrages de Jean-Paul Calvez. \cite{Calvez_1} \cite{Calvez_2} \\
Ce travail de conception à pour but de placer les étudiants dans un contexte industriel. Le cahier des charges fourni prend la forme d'un exemple réel où les spécifications du client sont exprimées.

