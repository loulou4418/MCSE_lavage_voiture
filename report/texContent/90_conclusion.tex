\section{Conclusion}



Pour conclure, la méthode MCSE apporte un outil complet offrant de nombreux avantages. 
Le principale étant d’effectuer une bonne structuration de sa conception, mais également de raisonner en faisant
abstraction de la technologie utilisée. 
Cela permet de ne pas se confiner dans une solution technique. 
Également, la partie spécification est très utile pour dégrossir le cahier des charges ou pour réduire la complexité de conceptions lourdes, d’éviter les erreurs. 
Cependant la méthode MCSE n’est pas applicable à des conceptions très basiques et simples. 
C’est en effet plus efficace de s’abstenir de cet outil car celui-ci est chronophage, et il est parfois nécessaire de réitérer l’ensemble du processus lorsqu’un problème de conception est identifié. 
Entre autres, l’outil qu’est la méthode MCSE est puissant mais difficile à maîtriser car celui-ci demande une grande expérience.
